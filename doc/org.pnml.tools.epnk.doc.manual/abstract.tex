\begin{quote}
\centerline{{\bf Abstract}}

The \emph{ePNK} is an Eclipse based framework and platform for developing
and integrating Petri net tools and applications. One of its core features
is that new \emph{Petri net types} can be plugged in, which does not require
any programming. A new Petri net type can be defined by providing a model
of its concepts, the so-called \emph{Petri net type definition} (\emph{PNTD}).
In addition, the ePNK allows adding new \emph{applications} on Petri nets
to the ePNK.

The ePNK builds on the concepts and models of the
\emph{Petri Net Markup Language} (\emph{PNML}), which is an XML based
interchange format for all kinds of Petri nets, which was published
as International Standard ISO/IEC~15909-2 in February 2011.
Technically, ISO/IEC 15909-2 is defining an interchange
format for three different kinds of high-level Petri nets and a simple version
of Place/Transition systems only. But, one of the objectives of PNML was to provide
a means for exchanging any kind of Petri net \cite{JKW00b,WeKi03,BCea03}. To this end, the concept of a 
\emph{Petri Net Type Definition} (\emph{PNTD}) was introduced, which is subject
of a newly issued standardisation project: ISO/IEC 15909-3.

There are many tools supporting one form of PNML or the other, and, in
particular, there is the PNML Framework \cite{DBLP:conf/apn/HillahKPT10}, which
helps tool developers to ease the implementation of PNML by providing a
framework and an API for loading and saving Petri net documents in PNML. This framework is
based on the \emph{Eclipse Modeling Framework} (\emph{EMF}) \cite{BSM06} and has
its focus on the underlying meta-models of Petri nets. The PNML Framework,
however, is not generic in the following sense: Whenever a new Petri net type is
created, the code for the complete tool needs to be regenerated.
Moreover, the PNML Framework does not come with a graphical editor
for Petri nets.

The ePNK overcomes these limitations: It provides an extension-point, so
that new Petri net types can be plugged in to the existing tool without
touching the code of the ePNK. For defining a new Petri net type, the developer,
basically, needs to create a class diagram defining the concepts of the new
Petri net type, along with a mapping of these concepts to XML syntax. This type can then
be plugged into the ePNK, and the graphical editor of the ePNK will be able
to edit nets of this new type with all its features. Likewise, the ePNK
allows to plug in new applications for the analysis, verification or
simulation of Petri nets. Moreover, it is possible to customize the
graphical representation of Petri nets and their specific features, and 
applications can visualize their results and interact with the user on top of
the graphic representation of the net in the graphical editor of the ePNK.

Actually, this was the idea when we started the development of the
\emph{Petri Net Kernel} (\emph{PNK}) about 20 years ago
\cite{KiDe96,Kin97d,KiWe01c}. At that time, however, we had to implement
the complete IDE functionality of the PNK ourselves. The ePNK is based
on Eclipse \cite{Eclipse-WWW}, so in the ePNK, we could focus on the
Petri net specific parts of such a tool. We get all the functionality
of a nice IDE, basically, for free. Therefore, we named the tool
\emph{ePNK} for \emph{Eclipse-based Petri Net Kernel}. But, it is only
the spirit and their idea that the PNK and the ePNK have in common;
technically, there is not a single line of code from the PNK in the ePNK,
and the ePNK is not compatible with the PNK.

What is more, we use the nice features of EMF, GMF, and Xtext
for developing the ePNK in a model-based way. In this way, the
complete development process of the ePNK is a case study in model-based
software engineering using EMF and related technologies. This, actually,
was the driving force behind this project.

This manual focuses on how
to use the ePNK as an end user, and on how a developer can use the
extension mechanisms of the ePNK for providing new Petri net types along
with their XML syntax, and how to add new applications to the ePNK.

A first version of this manual has been published in February 2011 as
IMM-Technical Report-2011-03 already, which referred to version 0.9.1 of the
ePNK. The second version of this document (IMM-Technical Reprot-2012-14)
referred to version 1.0 of the ePNK, which was released in October 2012. The
current version of this report refers to version 1.2 of the ePNK, which
was released in August 2017. It was updated with respect to the new features
of the ePNK and extended by a detailed tutorial, which discusses a
complete example in all technical details.
\end{quote}