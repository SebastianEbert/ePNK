%%%%%%%%%%%%%%%%%%%%%%%%%%%%%%%%%%%%%%%%%%%%%%%%%%%%%%%%%%%%%%%%%%%%%%%%%%%%%%%
%% Inside the ePNK                                                         %%
%%%%%%%%%%%%%%%%%%%%%%%%%%%%%%%%%%%%%%%%%%%%%%%%%%%%%%%%%%%%%%%%%%%%%%%%%%%%%%%
\chapter{Experience and outlook}
\label{chap:inside}

With version~1.0.0, the ePNK has reached a mature state and it should be
useful for end users who want to use its graphical editor for creating
PNML files and for using the simple function of the ePNK as they are.
The ePNK should be even more useful for developers who want to implement new
Petri net types and new functionality. In particular, it should be an ideal
platform for scientists who quickly want to test new functions and still want a
graphical editor that is nicely embedded to an IDE.
Actually, we use the ePNK ourselves for implementing the tool support for
the Event Coordination Notation (ECNO) \cite{Kin12b}.

Some of the plans for future extensions of the ePNK are discussed in
Sect.~\ref{sec:inside:plans}. Before discussing the future plans, we
briefly discuss the past in Sect.~\ref{sec:inside:experiences}:
the experiences with developing the ePNK in a model based way and in particular
with using EMF, GMF, and some related technologies

\section{Experiences with MBSE}
\label{sec:inside:experiences}

There are many Petri net tools out there already. Therefore, implementing
yet another one needed some additional motivation. When developing
the ePNK, this additional motivation was to gain some more experience with
the use of EMF and model-based development. To this end, we kept a
detailed log of how much time was spent on which parts of the development
(up to the first major release of version 0.9.1, the log accounts for
minutes).

Eventually, we might break down the time and experiences made in more detail.
Here we give an overview of the major steps and the rough time spent on
the major parts of the ePNK only:

\begin{description}
\item[40h] The (roughly) first 40~hours were spent on making a first model of
           the PNML core model and implementing the extensions necessary
           for plugging in new Petri net types and for hooking into the
           serialisation mechanisms of Eclipse and EMF, so that it could
           be configured -- and on using this new configuration mechanism for
           implementing the PNML syntax for serialisation instead of standard
           XMI. Of these 40~hours, about 10~hours were spent on debugging
           Eclipse's and EMF's serialisation mechanism in order to
           understand this serialisation mechanism, which unfortunately is
           not very well documented.
           
           After these first 40 hours, a basic version of the ePNK was working
           -- not supporting HLPNGs yet and without any graphical editor.
           
\item[20h] The next 20~hours were spent on implementing the framework
           for tool specific extensions -- and ignoring unknown tool specifc
           extensions, as well as on implementing the validation mechanisms
           and most of the constraints for the PNML core model\footnote
             {This concerns not only the constraints that are
             explicitly formulated as OCL constraints in the
             models of ISO/IEC~15909-2; this concerns all the constraints
             that are stated somewhere in the text of the standard,
             such as forbidden cycles between reference nodes, etc.}.
             
\item[50h] About 50~hours were spent on implementing HLPNGs, which includes
           making the models for most of the sorts and operators
           of HLPNGs, implementing a type system for checking correct
           typing, and for validating correctness, and implementing parsing
           and linking functions. These 50~hours include the time spent
           on adjusting the mechanisms of the ePNK so that it could deal
           with parsing and linking structured Petri net types. The
           parser itself is based on Xtext.
           
           Implementing a basic version of HLPNGs with only a few
           but representative sorts and operators took roughly 20~hours.
 
\item[80h] Implementing the GMF editor in a generic way and properly 
           integrating it with the EMF tree editor, and the parsing mechanisms
           for structured labels took about 80~hours. This time includes
           a lot of time investigating and experimenting with different
           options in achieving this.  
           
\item[30h] The remaining 30~hours were spent on implementing some functions
          as examples (used for the tutorials), on adding some of the
          last remaining sorts to the definition of HLPNGs, as well
          as on cleaning up the code and fixing some errors.           
\end{description}

Altogether, it took about 5 $\frac{1}{2}$ weeks working time (spent scattered
over about 7~month) to implement version~0.9.1 of the ePNK, which was the
first stable version of the ePNK. The core part of the ePNK was implemented
in 60~hours. About the same time went into implementing HLPNGs. The
implementation of the graphical editor took a major part of the time (80~hours).
The reason for that was that GMF itself was not made for building
generic editors, and we had to find ways of bringing genericity into GMF --
and we had to work around several GMF problems and quirks. Still, using GMF
might have saved us a significant amount of time considering the overall functionality
that we get for free by a GMF generated graphical editor -- and its smooth
integration with the Eclipse IDE.  

The overall experience was that the parts of the ePNK that concern EMF only
worked very smoothly and the use of EMF significantly sped up the
development process -- for a developer who has some experience with EMF
and some of its more advanced concepts already. Working with GMF was more
tedious and required much more experience and much more endurance --
using the debugger digging in the inner workings of GMF in order to find out
how some things work. This is partly due to the fact that GMF was lacking
the concepts for generic editors; but, genericity aside, GMF requires
much more experience than EMF in order to use it with benefit.

\section{Future plans}
\label{sec:inside:plans}

The ePNK can be and is used for different kinds of applications -- and it
makes it easy to quickly implement new Petri net types and new functions
and application. For making these functions more usable and also for
easing the fast creation of Petri nets with the ePNK editors, some
extension of the ePNK would be useful.
 
In this section, we give an overview of some extensions that are planned
for the ePNK in the future. These extensions concern ePNK's flexibility and
ease of use as well as some additional extension mechanisms. The order
indicates some priorities, and might be roughly the order in which the features are
implemented -- but, since nobody is paid for the work, it needs to be seen
how things turn out:
\begin{itemize}
  \item Right now, the ePNK does not ``know'' the functions and applications
        that are available for the ePNK. These are plugged in to the
        Eclipse platform as actions or commands -- and initiated by
        the Eclipse platform. It would be nice if there was an
        ePNK view that would, for a selected Petri net, show all
        the available functions and applications, which could be started
        from there by a single mouse click.
  
        To this end, an explicit extension point to plug in functions and
        applications for the ePNK needs to be defined. This extension point
        could also provide some means to give some meta information about
        the function or application, saying to which net types it applies
        and on its characteristics. 
        
        Such a plug-in mechanism along with a view for showing the
        functions and applications available for the selected Petri net
        will be implemented in a future version of the ePNK.
        
  \item Right now, the annotations are shown by marking the elements with
        a red overlay. Changing this is possible but quite complicated.
        
        A future version of the ePNK will support a mechanism for applications
        to provide a presentation descriptor, which defines how annotations
        should be shown (with some reasonable default implementation). And this
        descriptor should also be able to define how the end user can
        interact with the annotated elements and define actions that
        are initiated by that.
     
  \item Right now, adding all the needed labels to a Petri net element of
        a more complex net types such as HLPNGs with the ePNK editor is quite
        tedious: First, each label must be created, then the label must
        be connected to the element, and its type must be selected.
        
        In a future version of the ePNK, an action will be installed that,
        e.\,g.\ on a double click on a node, will add all the labels required by
        the Petri net type for that node. 

  \item Right now only the ePNK tree editor will show the correct ``dirty flag''
        after a change of the model. And the PNML document can be saved
        only from this tree editor.
  
        In a future version of the ePNK, the ``dirty flag'' will be updated in
        all editors that are open on that document and saving (in particular with CTRL-S)
        will work from any of these editors.

  \item Right now, the mapping of the model elements of the ePNK and of
        Petri net types to their XML representation must be programmed --
        if it needs to be changed. This means programming large tables,
        which is boring, tedious and error prone.
        
        In a future version of the ePNK, we might define an extension point for
        such XML mapping that directly takes a table instead of
        ``programming the table''.
        
  \item Right now, the ePNK comes with its own specific and fixedly
        implemented concrete syntax for the labels of HLPNGs. Since this
        syntax is not mandated by ISO/IEC~15909-2, different tools might
        use different concrete syntax for these labels. 
        
        It would be nice, if different versions of such concrete syntax for
        labels of HLPNGs could be plugged in to the ePNK, from which the user
        could choose.
          
  \item Up to now, the ePNK supports basic and structured PNML
        \cite{WeKi03} only. Modular PNML is not supported by the ePNK yet 
        -- nor is it part of ISO/IEC~15909-2:2011.
        
        In the long term, the ePNK should support also modular PNML
        (which was proposed in \cite{WeKi03} and some aspects worked
         out in more detail in \cite{Kin07} already). Implementing the
        EMF models would actually not be a big issue -- implementing
        the graphical editor of the ePNK so that it works for both
        PNML as of ISO/IEC~15909-2:2011 as well as for modular PNML
        is the actual challenge here -- and the reason for not
        having started on this endeavor yet.
\end{itemize}
  